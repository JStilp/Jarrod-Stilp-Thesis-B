\chapter{Introduction}
\label{cha:introduction}

\section{Background}
\label{sec:intro_background}

Within the healthcare industry, there is a continuous need for fast and reliable diagnostics of pathogens within patients. Identifying the presence of pathogens responsible for disease in a patient allows the appropriate preventive or corrective action to be taken and represents a crucial step in treating or preventing illness. This thesis was conducted for the benefit of and in collaboration with AusDiagnostics Pty Ltd.\\

Successful diagnostics, within the context of this thesis, can be summarised in three overall stages. Namely these are extraction, amplification and finally, analysis. This thesis concerns itself only with extraction. It should be noted that not all pathogen analysis and/or commercial diagnostic processes follow these steps strictly, however the processes and technologies applied by AusDiagnostics follow this procedure. The stages of this procedure may be summarised as follows:
\begin{enumerate}
	\item [1. Extraction] To begin the diagnosis, a clinical sample is obtained from the patient. This sample may consist of cerebrospinal fluid, faecal matter, urine or others, depending on the disease to be diagnosed. These samples contain the target DNA or RNA which will later be analysed to determine the presence of the pathogen and hence disease. They also contain a number of inhibitors to the process of amplification and analysis. Extraction is the process of removing said inhibitors and retaining only the target DNA or RNA. The result is referred to as a clean sample.
	\item  [2. Amplifcation] Amplification takes the clean sample and by one of many methods increases the overall count of the DNA. This may be with the intention of allowing multiple targets to be detected or to increase the sensitivity of the analysis.
	\item[3. Analysis] Analysis uses one of many available methods to search for the presence of biomarkers within the amplified clean sample. The presence of the biomarker indicates the result of the diagnosis.
\end{enumerate}

AusDiagnostics currently supplies customers with the instruments and chemical products required to complete stages two and three (amplification and analysis) of the diagnosis. This requires customers to purchase extraction equipment from alternative suppliers and represents a significant weakness and loss of profit. Research and development conducted by AusDiagnostics has determined that the optimal extraction method, when considering speed and efficiency, is super-paramagnetic bead based extraction. The beads utilised are of the shell-core variety. The core is composed of an iron oxide, which provides the super paramagnetic properties required for physical manipulation of the beads via a magnetic field. The shell is comprised of silica which, via chemical modification has the propensity to bond DNA and RNA to the bead surface. The techniques developed by researchers at AusDiagnostics utilising the magnetic silica beads have been validated and verified via manual operation. This thesis concerns itself with the automation of the developed extraction process, to produce a commercially viable robotic instrument. This instrument will be referred to as the Gene-Plex Extractor.\\

The extraction process to be automated has a number of notable requirements. These include liquid handling via precise pipetting, including mixing and liquid transfer. Also required is manipulation of the magnetic silica beads to separate the bonded and hence captured target DNA or RNA, along with heating to a specified, constant temperature to act to increase the rate of the chemical processes. The automation of the extraction process will be achieved by integrating the required capabilities into the robot produced by AusDiagnostics to conduct the amplification stage of diagnosis. The instrument, based of the Gene-Plex platform and sold as the High-Plex Processor, is pictured in it's current application in Figure \ref{fig:highplex}. The Gene-Plex platform is essentially a liquid handling robot. The robot carries out the amplification stage by precisely pipetting and transferring liquid mixtures between the tubes and instruments on the deck, using disposable tips. Each individual assay (an analysis conducted to determine the presence and amount of a substance within a volume) utilises an individual layout of components on the robots deck. This makes the platform highly configurable for differing setups.

\begin{figure}[!htb]
	\centering
	\includegraphics[width = \textwidth]{highplex.png}
	\caption[High-Plex Platform.]{The Gene-Plex liquid handling robot platform, as implemented as the High-Plex Processor.}
	\label{fig:highplex}
\end{figure}ˆ 
\FloatBarrier

\subsection{The Extraction Process}
\label{subsec:intro_extraction}

In order to allow the aim and scope of the work to be clearly defined, a condensed overview of the extraction process developed by AusDiagnostics is presented. It should be noted that unless explicitly stated, all operations are to be automated.\\

The extraction process requires that the clinical sample undergoes a number of chemical steps across different locations on the robot. To aid in understanding the liquid handling involved, Figure \ref{fig:decklayout} displays the important sites on the deck. The process will begin with the operator manually loading 24 individual clinical samples into the location labelled ``Clinical Samples". In order to conform to existing products used by customers, it is then required that the chemical processing takes place in the locations labelled ``Samples". In order to transfer the liquid between locations, the robot will pick up 1000$\mu$L tips from the locations marked ``Tips".\\

\begin{figure}[!htb]
	\centering
	\includegraphics[width = \textwidth]{decklayout.png}
	\caption[Extractor deck layout.]{The layout to be used for the Gene-Plex Extractor.}
	\label{fig:decklayout}
\end{figure}ˆ 
\FloatBarrier

The sample processing locations must accept modules that fit within the standard block size (SBS) format. These blocks are required to each accept 8 cassettes, one for each sample (pictured in Figure \ref{fig:cassette}). Each cassette includes 6 tubes, which will be the site of a particular chemical reaction in order to extract the target DNA and RNA:
\begin{enumerate}
	\item 500$\mu$L of clinical sample is pipetted into tube location 2, as marked in Figure \ref{fig:cassette}. Within this tube as supplied by AusDiagnostics, there will already be 10$\mu$L of magnetic bead mixture along with 440$\mu$L of lysis buffer. The lysis buffer will destroy the cell walls, releasing the target DNA or RNA to be bound to the magnetic silica beads. This process is required to take place at 60$\degree$C in order to increase the rate of reaction. Due to the low volume of liquid in this step, it is to be completed in one of the 1mL low profile tubes. This aids with liquid handling precision.
	\item The target DNA or RNA is now bound to the beads, which are suspended in the waste liquid (supernatant). In order to capture only the bead suspension, the entire liquid mixture is aspirated via the pipette tip and subsequently moved to the location marked ``Waste Separation". In this location, a magnetic field is to be applied to capture the magnetic silica beads within the pipette tip. While captured, the supernatant must then be expelled into a waste container in this location. The tip now contains only the magnetic beads with bound targets.
	\item Despite the supernatant having been expelled, there will still be a significant amount of waste retained on the bead surface. To remove this and clean the beads, a sequence of 3 wash steps must then occur in tube locations 1, 5 and 6. To achieve this, the beads are first re-suspended in the tube, which contains 800$\mu$L of a particular wash buffer. The suspension must then be mixed via pipetting in order to to ensure the bead surface is properly exposed. The entire liquid mixture is then aspirated once again and transferred to the waste location, where via the same method as step 2, the waste liquid is disposed of while the beads are retained. This process is repeated 3 times in the tubes noted above to ensure no waste matter clings to the bead surface.
	\item With the beads now clean and still bound to the target, the mixture is transferred to location 3 of the cassette. The elution buffer is used to break the bond between the target DNA and RNA and the magnetic silica beads. This process is also required to be completed at 60$\degree$C to reduce the time required. Following this step, the target DNA or RNA of interest is now contained within the elution buffer, along with the magnetic silica beads which are now waste.
	\item The final step is to capture only the target DNA or RNA within the elution buffer, leaving behind the magnetic beads. This is required to take place in tube location 4. This is a low volume reaction, containing only 100$\mu$L of elution buffer and is therefore completed in the final small profile tube. The pipette must then aspirate this mixture, following bead removal, and contain only the elution buffer and target mixture required for amplification in stage 2 of the assay.
\end{enumerate}

\begin{figure}[!htb]
	\centering
	\includegraphics{cassette.png}
	\caption[Extraction cassette tubes.]{The tube format to be used as the site of the sample processing.}
	\label{fig:cassette}
\end{figure}ˆ 
\FloatBarrier

\section{Aim}
\label{sec:intro_aim}
%May bee to summarise what is missing from the robot before stating the aim
This work aims to develop and integrate into the Gene-Plex Robot Platform the hardware necessary for carrying out the described chemical processes, the required magnetic manipulations and the controller necessary to maintain the stipulated constant 60$\degree$C temperature.

\section{Scope}
\label{sec:intro_scope}
The features required by the Gene-Plex Extractor, over those already part of the Gene-Plex Processor, can be grouped into 3 overall categories. These are detailed below, including a clear definition of their inclusion or exclusion from the scope of this work.

\subsection{Mechanical}
\label{sec:intro_mechanical}
The summary of the extraction process to be implemented, given in Section \ref{subsec:intro_extraction}, revealed a number of missing mechanical components when compared to the description of the Gene-Plex Processor given in Section \ref{sec:intro_background}. 

\begin{enumerate}
	\item [SBS Processing Module] As was noted in Section \ref{subsec:intro_extraction}, the extraction process is to take place within cassettes consisting of 6 tubes. These must be located in the 3 SBS deck locations as marked in Figure \ref{fig:decklayout}, with each SBS location accepting 8 cassettes. The design of the module which will accept these cassettes is within the scope of this thesis and will form a large component of the design work undertaken.

	\item [Heating Element] In order to create a commercially viable product, it has been stipulated that the tubes within each cassette where the the lysis buffers and elution buffers are held (tubes 2 and 3), a temperature of 60$\degree$C must be maintained for the duration of processing. This is a core requirement of the extraction process and is included in the scope of this project.
	
	%NEED a consistent way of refering to tubes. Here tubes 2 and 4 actually refer to steps 2 and 4. Show a diagram of this.

	\item [Magnetic Separation] The Gene-Plex Extractor can be seen to require magnetic manipulation of the silica beads in two different locations. In order to allow the extracted DNA or RNA to be captured after separation from the beads in tube 4, the beads must be retained in the tube while the liquid is aspirated by the pipette. Secondly, the magnetic beads must be captured within the pipette tip during the expulsion of the waste supernatant after all wash steps are completed. The conceptual design of the hardware for magnetic separation required in the process of tube 4 bead retention is defined to be within the scope of this work. The capture of beads in the pipette tip however, is not. The scope of this portion of work is limited to conceptual design due to AusDiagnostics' recent push for ISO certification. Due to these efforts, the validation and verification process of this component must meet a large number of requirements and must use input from a range of AusDiagnostics experts. Therefore, validation and verification is not within the scope of this work.

	\item [Waste Disposal] Following the wash steps and the subsequent separation of the magnetic beads and the bound DNA or RNA from the wash buffer, the supernatant must be hygienically disposed of. This waste disposal method is required in order to ensure that no contamination occurs between the clinical samples being processed. It is also crucial to ensure that no contact can occur between the operator and the biological matter which is a by product of the extraction process. Due to the demands of ISO certification process involving the handling of biological materials, this portion of work is excluded from the scope of this work.
	
\end{enumerate}

\subsection{Electronic}
\label{sec:intro_electronics}
In order to account for the newly integrated capabilities of the Gene-Plex Extractor, some modifications to the Gene-Plex Processor electronics are required. Due to the temperature control requirements of the extraction process, an electronics board capable of communicating with the robot software and driving the heating elements appropriately is required. This however is not included in the scope of this work for a number of reasons. One of the components of the Gene-Plex processor is a device called the MTX Cycler. This device controls the temperature of the already extracted sample (during stage 2, amplification) precisely between set temperatures to enable PCR (Polymerase Chain Reaction) to occur and hence amplify the sample. The MTX Cycler controls the temperature of liquid volumes significantly smaller than those involved in extraction and in a different form factor, making it unsuitable for this application. It does however provide the necessary electronics and interfaces required to drive any standard form of heating element used here. Furthermore, the electronic board used to control the MTX Cycler is currently being integrated into a newly developed control board for the Gene-Plex Platform by AusDiagnostics engineers. This new board will also be capable of driving the heating elements selected for this application. Therefore, due to the redundancy of work on this component of the Gene-Plex Extractor and due to a suitable requirement existing, the development of the required electronic is not within the scope of this work.

\subsection{Software}
\label{sec:intro_software}
In order to enable the Gene-Plex Platforms controlling software to utilize the newly integrated capabilities, two main components of software must be created:
\begin{enumerate}
	\item[Temperature Controller] In order to control the heating elements to maintain a stable and accurate 60$\degree$C for the lysis and elution stages, a temperature controller must be implemented. This temperature controller must be designed according to the temperature response of the heated hardware using appropriate methods and implemented as a controller in software. This element of the Gene-Plex Extractor's requirements will also form a core part of this work and is included within the project scope.
	\item [Routine Addition] As was briefly noted in Section \ref{sec:intro_background}, the Gene-Plex Platform is highly adaptable in its ability to perform assays requiring differing liquid handling operations. The platform achieves this by utilizing a number of library routines which may be called upon the assay requiring a particular movement, such as expelling liquid, moving to a certain robot position etc. While the vast majority of these library routines will be directly effective within the Gene-Plex Extractor, due to the newly integrated capabilities, further routines will be required. These will include routines for the mixing of liquids in the cassette tubes, magnetic separation and a table of definitions which specifies the locations of each of the tubes and pieces of deck hardware. These software additions are not included within the scope of this work and will be implemented by software engineers at AusDiagnostics.
\end{enumerate}