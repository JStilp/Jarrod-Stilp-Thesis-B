\chapter{Discussion}
\label{cha:discussion}

% The discussion is where you really use your intuition to explain the results.
%
% At this stage it's usually a good idea to read through your thesis from the 
% start and check everything flows logically. Check each chapter has a strong
% introduction and conclusion. This will reinforce in your mind what you are
% doing.
%
% I like to think of the discussion as doing four things:
% 1. Structure and summarise your results into key points. And then for each...
% 2. Explain what happened
% 3. Discuss why this happened
% 4. Lead the reader toward why this significant / what this means
%
% In your conclusions you will draw from each of these points.
%
% Evans & Gruba [1] acknowledge that the discussion is a particularly hard chapter
% to start writing. They present a 4-step method to help you out.
% 1. List all the things you know that you didn't know when you started. This
%    should be a long list and doesn't have to answer the aim or have any 
%    form of structure. General ideas, snippets of knowledge, insights, things
%    that went wrong, anything!
% 2. Group these things in a rational, stuctured way
% 3. Give each group a name. This formalises it's purpose and you can use it
%    as a section heading if you like.
% 4. Structure the ideas within each group in a logical way. Reject any which
%    you don't think are beneficial at this stage.
%
% [1] Evans, D., Gruba, P. & Zobel, J., 2011. How to Write a Better Thesis, Melbourne University Press.
