\chapter{Results}
\label{cha:results}
% This chapter should display the results of your experiments. It should be entirely
% factual, so leave the discussion you draw from it for the Discussion chapter.
%
% You may choose to structure your thesis slightly differently, but the overall 
% approach should be the same.
%
% 1. By now you should have already described your methodology and  evaluation 
%    procedures. There is no reason to discuss these in the results.
% 2. Do not include raw data in the results chapter, leave it for the appendix 
%    or simply don't include it. If you don't think people will refer to it it's
%    probably not worth including.
% 3. Display your results in an informative way. Charts, tables, summaries, etc.
%
% If your results form a complete chapter you can use the conclusion section to
% state how the results answer your hypothesis. If you don't have a spearate
% conclusion chapter make sure you answer the hypothesis as you go. The reader
% should have answers to these before the Discussion.

The problem at hand along with the need for a solution are now well defined. Paired with an understanding of important previous works in this area and the technologies available, the design of each of the components may be given consideration. This chapter will present the designs of the required additions to the Gene-Plex Extractor, as were defined in Section \ref{sec:intro_scope}, Scope. This will include relevant aspects of the design process, the conducted simulations along with validations and verifications via experimental methods. Through the sections below, the reader will be familar with the design of each of the components along with the resulting performance of the realised and implemented component.\\

\section{Processor Module Hardware}

\subsection{Requirements}

\subsection{Design}

\subsection{Simulation}

\section{Temperature Controller}

\subsection{Requirements}

\subsection{Design}

\subsection{Simulation}

\section{Processor Module}

\subsection{Validation}

\subsection{Verification}

\section{Magnetic Separation Station}